%!TEX root = pfc-memoria.tex
%!TEX encoding = UTF-8 Unicode

\chapter{Análisis de requisitos}

\flushbottom

Este capítulo contiene la Especificación de Requisitos del Sistema (ERS en español, ó SRS del inglés \emph{System Requirements Specification})\index{ERS}\index{SRS} siguiendo las directrices de la norma IEEE~Std~830-1998 \citep{std830-1998} usando la notación de UML~2.5 \citep{UML2.5}

\section{Actores del sistema}

\SRSActor{title={Usuario},
label=usuario,
desc={Usuario normal de la aplicación, comúnmente un alumno en un puesto informático con la aplicación gráfica instalada, pantalla, procesador, teclado y ratón.},
comments={-/-},
stability=Alta}

\section{Objetivos}

\SRSObj{title={Biblioteca unificada de NLP y ML},
label=biblioteca-nlp-ml,
desc={Desarrollar una biblioteca unificada de NLP y ML para el análisis del sentimiento o polaridad de opiniones. La biblioteca debe proporcionar la funcionalidad necesaria para:
\begin{enumerate}[a)]
\item procesar texto mediante diversos filtros (en la etapa de entrenamiento y la de clasificación),
\item aprender a clasificar el sentimiento automáticamente mediante algoritmos parametrizables de aprendizaje automático.
\end{enumerate}
Ambos grupos de funcionalidad deberán permitir la colaboración entre sí, en secuencia.},
comments={-/-},
stability=Alta}

\newpage
\SRSObj{title={GUI para el análisis de sentimiento},
label=gui,
desc={Desarrollar una aplicación de escritorio, con Interfaz Gráfica de Usuario multiplataforma, que haga uso del \refSRSObj{biblioteca-nlp-ml} para guiar al alumno en el análisis del sentimiento de opiniones sin necesidad de codificar nada.},
comments={-/-},
stability=Alta}


\section{Requisitos de información}

\SRSIrq{title={Fichero de configuración de sesión},
label=yaml,
desc={El sistema deberá proporcionar una manera de guardar y recuperar un fichero con la información necesaria para establecer la configuración de opciones de la sesión de trabajo actual con el programa.},
comments={Fichero de entrada y de salida.
Se recomienda un formato consumible por máquina y humano, sencillo, de tipo YAML.},
objasoc={\fullrefSRSObj{gui}},
reqasoc={-/-},
data={Datos de los parámetros ajustables presentes en los elementos del GUI:
\begin{itemize}
\item ruta a los ficheros de entrada,
\item opciones activas de preprocesamiento de texto,
\item opciones de la extracción de características,
\item parámetros establecidos en cada método de aprendizaje automático,
\item puntuaciones relativas obtenidas (valor-F),\index{valor-F}
\item cualquier otro dato necesario para poder repetir el experimento y obtener los mismos resultados.
\end{itemize}},
priority=Media,
stability=Alta}

\newpage
\SRSIrq{title={Formato del fichero de entrenamiento},
label=traintsv,
desc={Es un fichero de texto plano, en formato TSV (tab-separated values), con línea de cabeceras y de cuatro columnas, sin adorno de los literales de cadena.},
comments={Fichero de entrada. El detalle de los formatos de fichero se encuentra en \url{https://www.kaggle.com/c/sentiment-analysis-on-movie-reviews/data}},
objasoc={\fullrefSRSObj{gui}},
reqasoc={-/-},
data={Las columnas son: \begin{enumerate}
\item PhraseId (numérico y único)
\item SentenceId (numérico)
\item Phrase (cadena con el texto)
\item Sentiment (numérico).
\end{enumerate}
La clasificación del sentimiento corresponde de la siguiente manera:
\begin{eqnarray*}
0 & \longrightarrow & \text{negativo} \\
1 & \longrightarrow & \text{algo negativo} \\
2 & \longrightarrow & \text{neutro} \\
3 & \longrightarrow & \text{algo positivo} \\
4 & \longrightarrow & \text{positivo}
\end{eqnarray*}},
priority=Alta,
stability=Alta}

\SRSIrq{title={Formato del fichero de evaluación},
label=testtsv,
desc={Es un fichero de texto plano, en formato TSV (tab-separated values), con línea de cabeceras y de tres columnas, sin adorno de los literales de cadena.},
comments={Fichero de entrada. El detalle de los formatos de fichero se encuentra en \url{https://www.kaggle.com/c/sentiment-analysis-on-movie-reviews/data}},
objasoc={\fullrefSRSObj{gui}},
reqasoc={\fullrefSRSIrq{traintsv}},
data={Análogo al \refSRSIrq{traintsv}, pero sin la última columna del sentimiento.},
priority=Alta,
stability=Alta}

\newpage
\SRSIrq{title={Formato del fichero de envío},
label=submissioncsv,
desc={Es un fichero de texto plano, en formato CSV (comma-separated values), con línea de cabeceras y de dos columnas},
comments={Fichero de salida. El detalle de los formatos de fichero se encuentra en \url{https://www.kaggle.com/c/sentiment-analysis-on-movie-reviews/data}},
objasoc={\fullrefSRSObj{gui}},
reqasoc={\fullrefSRSIrq{traintsv}},
data={Las columnas son:\begin{enumerate}
\item PhraseId (numérico y único)
\item Sentiment (numérico). Análoga correspondencia que en \refSRSIrq{traintsv}.
\end{enumerate}},
priority=Media,
stability=Alta}

\newpage
\section{Requisitos funcionales}

Se puede ver un diagrama completo de casos de uso en la \autoref{fig:uml-uc} (página~\pageref{fig:uml-uc}).

\subsection{Grupo funcional E/S}

\SRSUc{title={Cargar fichero de entrenamiento},
label=loadtraintsv,
desc={Mediante esta función, el usuario cargará un fichero de entrenamiento (\path{train.tsv}).},
precond={-/-},
steps={\begin{enumerate}
\item El usuario solicita cargar un nuevo fichero de entrenamiento.
\item El sistema ofrece un cuadro de diálogo para que el usuario seleccione el fichero.
\item El usuario selecciona el fichero.
\item El sistema ofrece la posibilidad de cargar sólo un número concreto de las primeras líneas.
\item El usuario escoge si se carga sólo un número concreto de las primeras líneas, o todas las del fichero.
\item El sistema lee el fichero a memoria y muestra el contenido del mismo en la tabla de resultados.
\end{enumerate}},
postcond={\begin{enumerate}
\item Se asocia en el diccionario de configuración la ruta al fichero de entrenamiento y la opción de carga parcial de las primeras líneas.
\item El contenido del fichero se encuentra visible de forma tabulada en la tabla de datos.
\end{enumerate}},
exceptions={Si el formato de fichero no concuerda con el especificado en \refSRSIrq{traintsv}, se produce un error.},
comments={-/-},
objasoc={\fullrefSRSObj{gui}},
reqasoc={\fullrefSRSIrq{traintsv}},
priority=Alta,
stability=Alta}

\begin{landscape}
\begin{figure}[htbp]
\centering
\resizebox{!}{0.76\textwidth}{\input{use-cases.puml}}
\caption{Diagrama de casos de uso}
\label{fig:uml-uc}
\end{figure}
\end{landscape}

%\begin{figure}[htbp]
%\centering
%\resizebox{0.6\textwidth}{!}{\input{use-cases-es.puml}}
%\caption{Diagrama de casos de uso (E/S)}
%\label{fig:uml-uc-es}
%\end{figure}
%
%\begin{figure}[htbp]
%\centering
%\resizebox{0.6\textwidth}{!}{\input{use-cases-sentanal.puml}}
%\caption{Diagrama de casos de uso (Análisis del senimiento)}
%\label{fig:uml-uc-sentanal}
%\end{figure}

\SRSUc{title={Cargar fichero de evaluación},
label=loadtesttsv,
desc={Mediante esta función, el usuario cargará un fichero de evaluación (\path{test.tsv}).},
precond={-/-},
steps={\begin{enumerate}
\item El usuario solicita cargar un nuevo fichero de evaluación.
\item El sistema ofrece un cuadro de diálogo para que el usuario seleccione el fichero.
\item El usuario selecciona el fichero.
\item El sistema ofrece la posibilidad de cargar sólo un número concreto de las primeras líneas.
\item El usuario escoge si se carga sólo un número concreto de las primeras líneas, o todas las del fichero.
\item El sistema lee el fichero a memoria y muestra el contenido del mismo en la tabla de resultados.
\end{enumerate}},
postcond={\begin{enumerate}
\item Se asocia en el diccionario de configuración la ruta al fichero de evaluación y la opción de carga parcial de las primeras líneas.
\item El contenido del fichero se encuentra visible de forma tabulada en la tabla de datos.
\end{enumerate}},
exceptions={Si el formato de fichero no concuerda con el especificado en \refSRSIrq{testtsv}, se produce un error.},
comments={-/-},
objasoc={\fullrefSRSObj{gui}},
reqasoc={\fullrefSRSIrq{testtsv}},
priority=Alta,
stability=Alta}

\newpage
\SRSUc{title={Cargar fichero de configuración de sesión},
label=loadyaml,
desc={Mediante esta función, el usuario cargará un fichero de configuración de la sesión.},
precond={El fichero de configuración seleccionado habrá sido previamente creado mediante \refSRSUc{saveyaml}, o escrito manualmente siguiendo un formato compatible.},
steps={\begin{enumerate}
\item El usuario solicita cargar un fichero de configuración de sesión.
\item El sistema ofrece un cuadro de diálogo para que el usuario seleccione el fichero.
\item El usuario selecciona el fichero.
\item El sistema lee el fichero de configuración de sesión y establece todos los parámetros del diccionario de configuración actual a los valores especificados en el fichero de configuración de sesión seleccionado, actualizando los controles del GUI de forma consistente; y se muestra activa la primera pestaña del flujo de trabajo.
\end{enumerate}},
postcond={\begin{enumerate}
\item Se asocia en el diccionario de configuración todos parámetros y valores indicados en el fichero.
\item Se actualiza el GUI para reflejar estos cambios, y se activa la primera pestaña del flujo de trabajo.
\end{enumerate}},
exceptions={Si el formato de fichero no concuerda con el generado mediante \refSRSUc{saveyaml}, se produce un error.},
comments={-/-},
objasoc={\fullrefSRSObj{gui}},
reqasoc={\mbox{\fullrefSRSIrq{yaml},} \makebox[10em]{} \mbox{\fullrefSRSUc{saveyaml}}},
priority=Alta,
stability=Alta}

\newpage
\SRSUc{title={Salvar fichero de configuración de sesión},
label=saveyaml,
desc={Mediante esta función, el usuario creará y guardará un fichero de configuración de la sesión actual.},
precond={-/-},
steps={\begin{enumerate}
\item El usuario solicita savar un fichero de configuración de sesión.
\item El sistema ofrece un cuadro de diálogo para que el usuario indique el nombre de fichero.
\item El usuario selecciona el fichero.
\item El sistema lee todos los parámetros con sus valores actuales del diccionario de configuración y los escribe en un nuevo fichero con la ruta y nombre seleccionados.
\end{enumerate}},
postcond={\begin{enumerate}
\item Se ha creado o actualizado el fichero con los valores de la configuración de la sesión actual; capaz de ser interpretado por \refSRSUc{loadyaml}.
\end{enumerate}},
exceptions={Si el fichero ya existía, se pregunta si se desea sobreescribir o no; en cuyo negativo caso el sistema solicitará un nuevo nombre único del fichero a crear.},
comments={-/-},
objasoc={\fullrefSRSObj{gui}},
reqasoc={\mbox{\fullrefSRSIrq{yaml},} \makebox[10em]{} \mbox{\fullrefSRSUc{loadyaml}}},
priority=Alta,
stability=Alta}

\newpage
\SRSUc{title={Salvar fichero de resultado de la clasificación},
label=savesubmissioncsv,
desc={Mediante esta función, el usuario creará y guardará un fichero de resultado de la clasificación predicha para el conjunto de evaluación.},
precond={El sistema ha clasificado automáticamente el conjunto de evaluación.},
steps={\begin{enumerate}
\item El usuario solicita salvar el fichero de resultado de la clasificación.
\item El sistema ofrece un cuadro de diálogo para que el usuario indique el nombre de fichero.
\item El usuario selecciona el fichero.
\item El sistema produce un fichero de resultado de la clasificación siguiendo el formato especificado en \refSRSIrq{submissioncsv}.
\end{enumerate}},
postcond={Se ha creado o actualizado el fichero con el resultado de la clasificación.},
exceptions={Si el fichero ya existía, se pregunta si se desea sobreescribir o no; en cuyo negativo caso el sistema solicitará un nuevo nombre único del fichero a crear.},
comments={-/-},
objasoc={\fullrefSRSObj{gui}},
reqasoc={\fullrefSRSIrq{submissioncsv}},
priority=Alta,
stability=Alta}

\newpage
\subsection{Grupo funcional Análisis del sentimiento}

\SRSUc{title={Preprocesar texto},
label=preproc,
desc={El usuario establece opciones de procesamiento del texto de los documentos de entrada y el sistema aplica los filtros indicados sobre los datos.},
precond={El fichero de entrenamiento ha sido cargado haciendo uso del \refSRSUc{loadtraintsv}.},
steps={\begin{enumerate}
\item El usuario selecciona la pestaña de preprocesamiento de texto.
\item El sistema ofrece un conjunto de opciones y filtros ajustables para el procesamiento: a) volver a unir contracciones, b) expandir contracciones, c) suprimir palabras vacías d) reemplazar con la raíz e) reemplazar con el lema, f) anotar la parte-de-la-oración
\item El usuario establece los parámetros deseados y confirma.
\item El sistema modifica los documentos de entrada aplicando los filtros indicados y actualiza la tabla de datos de resultados.
\end{enumerate}},
postcond={\begin{enumerate}
\item El sistema ha actualizado el diccionario de configuración de sesión con los parámetros indicados.
\item El sistema ha procesado el texto de los documentos de entrada y actualizado la tabla de resultados.
\end{enumerate}},
exceptions={-/-},
comments={-/-},
objasoc={\fullrefSRSObj{gui}},
reqasoc={-/-},
priority=Alta,
stability=Alta}

\newpage
\SRSUc{title={Extraer características},
label=features,
desc={El usuario establece opciones de extracción y selección de características del texto de los documentos de entrada preprocesados y el sistema vectoriza las características del conjunto de entramiento.},
precond={El conjunto de entrenamiento ha sido preprocesado mediante el \refSRSUc{preproc}.},
steps={\begin{enumerate}
\item El usuario selecciona la pestaña de extracción de características.
\item El sistema ofrece un conjunto de opciones de extracción y selección de características: a) extracción de conjuntos de $n$-gramas con \mbox{$1\leq n\leq 20$}, b) $n$-gramas con un mínimo de frecuencia documental absoluta o relativa en el conjunto de entrada, c) $n$-gramas con un máximo de frecuencia documental absoluta o relativa en el conjunto de entrada, d) número de características más significativas, e) eliminar características con poca varianza.
\item El usuario establece los parámetros deseados y confirma.
\item El sistema extrae y selecciona las características siguiendo los parámetros indicados, vectoriza cada documento y actualiza la tabla de resultados con una columna por cada característica.
\end{enumerate}},
postcond={\begin{enumerate}
\item El sistema ha actualizado el diccionario de configuración de sesión con los parámetros indicados.
\item El sistema ha realizado la extracción y selección de características teniendo en cuenta los parámetros indicados, actualizado la información en la tabla de datos con los vectores de características de los documentos.
\end{enumerate}},
exceptions={Si el texto de los documentos de entrenamiento no ha sido preprocesado por el \refSRSUc{preproc}, se utilizará el conjunto de entrenamiento de manera literal y original, sin alterar usando el resultado del \refSRSUc{loadtraintsv}; como punto de partida para la extracción y selección de características.},
comments={-/-},
objasoc={\fullrefSRSObj{gui}},
reqasoc={-/-},
priority=Alta,
stability=Alta}

\newpage
\SRSUc{title={Iniciar aprendizaje automático},
label=learn,
desc={El usuario establece opciones de aprendizaje automático y el sistema entrena el modelo de aprendizaje seleccionado, autoevaluándose con un porcentaje de documentos sobre el tamaño de entrenamiento.},
precond={El conjunto de entrenamiento ha sido vectorizado en características mediante el \refSRSUc{features}.},
steps={\begin{enumerate}
\item El usuario selecciona la pestaña de aprendizaje automático.
\item El sistema ofrece un conjunto de opciones de aprendizaje: a) porcentaje de división del conjunto de entrenamiento en partes de entrenamiento y de autoevaluación, b) opción para re-dividir los subconjuntos, c) selección del modelo de entrenamiento, d) opciones del modelo de entrenamiento.
\item El usuario establece los parámetros deseados y confirma.
\item El sistema entrena el modelo deseado de aprendizaje usando la porción indicada de documentos vectorizados en características y autoevaluándose con la porción restante.
\end{enumerate}},
postcond={\begin{enumerate}
\item El sistema ha actualizado el diccionario de configuración de sesión con los parámetros indicados.
\item El sistema ha entrenado el modelo indicado y se ha autoevaluado usando la predicción proporcionada por el modelo, indicando la puntuación (valor-F).\index{valor-F}
\end{enumerate}},
exceptions={Si los parámetros indicados no son apropiados (insuficientes, incoherentes o fuera de rango) para el modelo y el subconjunto de prueba concretos, se muestra un error.},
comments={-/-},
objasoc={\fullrefSRSObj{gui}},
reqasoc={-/-},
priority=Alta,
stability=Alta}

\newpage
\SRSUc{title={Clasificar conjunto de evaluación},
label=classify,
desc={El sistema clasifica el conjunto de evaluación indicado por el usuario.},
precond={Al menos uno de los modelos de aprendizaje automático ha sido entrenado durante la sesión actual, mediante el \refSRSUc{learn}.},
steps={\begin{enumerate*}
\item El usuario selecciona la pestaña de clasificación.
\item Se realiza el \fullrefSRSUc{loadtesttsv}.
\item El sistema habilita el botón de preprocesamiento.
\item El usuario pulsa el botón de preprocesamiento.
\item El sistema aplica los mismos filtros de preprocesamiento que se usaron durante el entrenamiento, pero aplicado al conjunto de evaluación; actualiza la tabla de datos mostrando el texto de los documentos preprocesados, y habilita el botón de extracción de caracteristicas.
\item El usuario pulsa el botón de extracción de características.
\item El sistema aplica las misma opciones de extracción y selección de características que se usaron durante el entrenamiento, pero aplicados al conjunto de evaluación; actualiza la tabla de datos mostrando los vectores de características, y habilita el botón de calificación y la selección de modelo de aprendizaje.
\item El usuario elige un modelo de aprendizaje y pulsa el botón de clasificación.
\item El sistema clasifica los documentos de evaluación vectorizados mediante la predicción proporcionada por el modelo de aprendizaje entrenado; actualiza la tabla de datos con el formato de salida necesario (\refSRSIrq{submissioncsv}) con el sentimiento predicho; y habilita el botón de salvar fichero de resultados de la clasificación.
\item Se realiza el \fullrefSRSUc{savesubmissioncsv}.
\end{enumerate*}},
postcond={\begin{enumerate*}
\item El sistema ha actualizado el diccionario de configuración de sesión con los parámetros indicados.
\item El sistema ha entrenado el modelo indicado y se ha autoevaluado usando la predicción proporcionada por el modelo, indicando la puntuación (valor-F).\index{valor-F}
\end{enumerate*}},
exceptions={El usuario puede abortar el proceso en cualquier momento, volver atrás y empezar de nuevo, sin obligación de completar exhaustivamente el flujo de secuencia normal especificado.},
comments={-/-},
objasoc={\fullrefSRSObj{gui}},
reqasoc={\mbox{\fullrefSRSIrq{submissioncsv},} \makebox[8em]{} \mbox{\fullrefSRSUc{loadtesttsv},} \makebox[8em]{} \mbox{\fullrefSRSUc{savesubmissioncsv}}},
priority=Media,
stability=Alta}

\newpage
\section{Requisitos no funcionales}

\SRSNfr{title={Multiplataforma},
label=multiplataforma,
desc={La aplicación será multiplataforma, al menos para los sistemas Mac~OS~X, Windows y GNU/Linux.},
comments={La estética del GUI no es determinante.},
objasoc={\fullrefSRSObj{gui}},
reqasoc={-/-},
priority=Alta,
stability=Alta}

\SRSNfr{title={Con visualización de la tabla de datos},
label=tabla-datos,
desc={En la medida de lo posible, debe verse una tabla de datos para examinar cómo se van aplicando las sucesivas transformaciones.},
comments={Es para promover el uso didáctico de la aplicación.},
objasoc={\fullrefSRSObj{gui}},
reqasoc={-/-},
priority=Media,
stability=Alta}

\SRSNfr{title={Con visualización del progreso},
label=progreso,
desc={En la medida de lo posible, debe verse una indicación del progreso en cada función que se realice sobre los datos.},
comments={Es para no dar la impresión al usuario de que la aplicación se haya bloqueado, al tratar con tablas de datos grandes.},
objasoc={\fullrefSRSObj{gui}},
reqasoc={-/-},
priority=Media,
stability=Alta}

\raggedbottom
