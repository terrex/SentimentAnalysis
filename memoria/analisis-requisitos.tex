%!TEX root = pfc-memoria.tex
%!TEX encoding = UTF-8 Unicode

\chapter{Análisis de requisitos}

Este capítulo contiene la Especificación de Requisitos del Sistema (ERS en español, ó SRS del inglés \emph{System Requirements Specification})\index{ERS}\index{SRS} siguiendo las directrices de la norma IEEE~Std~830-1998 \citep{std830-1998} usando la notación de UML~2.5 \citep{UML2.5}

\section{Actores del sistema}

\SRSActor{title={Usuario},
label=usuario,
desc={Usuario normal de la aplicación, comúnmente un alumno en un puesto informático con la aplicación gráfica instalada, pantalla, procesador, teclado y ratón.},
comments={-/-},
stability=Alta}

\section{Objetivos}

\SRSObj{title={Biblioteca unificada de NLP y ML},
label=biblioteca-nlp-ml,
desc={Desarrollar una biblioteca unificada de NLP y ML para el análisis del sentimiento o polaridad de opiniones. La biblioteca debe proporcionar la funcionalidad necesaria para
\begin{enumerate}[a)]
\item procesar texto mediante diversos filtros (en la etapa de entrenamiento y la de clasificación),
\item aprender a clasificar el sentimiento automáticamente mediante algoritmos parametrizables de aprendizaje automático.
\end{enumerate}
Ambos grupos de funcionalidad deberán permitir la colaboración entre sí, en secuencia.},
comments={-/-},
stability=Alta}

\newpage
\SRSObj{title={GUI para el análisis de sentimiento},
label=gui,
desc={Desarrollar una aplicación de escritorio, con Interfaz Gráfica de Usuario multiplataforma, que haga uso del \refSRSObj{biblioteca-nlp-ml} para guiar al alumno en el análisis del sentimiento de opiniones sin necesidad de codificar nada.},
comments={-/-},
stability=Alta}


\section{Requisitos de información}

\SRSIrq{title={Fichero de configuración de sesión},
label=yaml,
desc={El sistema deberá proporcionar una manera de guardar y recuperar un fichero con la información necesaria para establecer la configuración de opciones de la sesión de trabajo actual con el programa.},
comments={Se recomienda un formato consumible por máquina y humano, sencillo, de tipo YAML.},
objasoc={\fullrefSRSObj{gui}},
reqasoc={-/-},
data={Datos de los parámetros ajustables presentes en los elementos del GUI:
\begin{itemize}
\item ruta a los ficheros de entrada,
\item opciones activas de preprocesamiento de texto,
\item opciones de la extracción de características,
\item parámetros establecidos en cada método de aprendizaje automático,
\item puntuaciones relativas obtenidas ($f$-values),
\item cualquier otro dato necesario para poder repetir el experimento y obtener los mismo resultados.
\end{itemize}},
priority=Media,
stability=Alta}

\newpage
\SRSIrq{title={Formato del fichero de entrenamiento},
label=traintsv,
desc={Es un fichero de texto plano, en formato TSV (tab-separated values) de cuatro columnas, sin adorno de los literales de cadena.},
comments={El detalle de los formatos de fichero se encuentra en \url{https://www.kaggle.com/c/sentiment-analysis-on-movie-reviews/data}},
objasoc={\fullrefSRSObj{gui}},
reqasoc={-/-},
data={Las columnas son: \begin{enumerate}
\item PhraseId (numérico y único)
\item SentenceId (numérico)
\item Phrase (cadena con el texto)
\item Sentiment (numérico).
\end{enumerate}
La clasificación del sentimiento corresponde de la siguiente manera:
\begin{eqnarray*}
0 & \longrightarrow & \text{negativo} \\
1 & \longrightarrow & \text{algo negativo} \\
2 & \longrightarrow & \text{neutro} \\
3 & \longrightarrow & \text{algo positivo} \\
4 & \longrightarrow & \text{positivo}
\end{eqnarray*}},
priority=Alta,
stability=Alta}

\SRSIrq{title={Formato del fichero de evaluación},
label=testtsv,
desc={Es un fichero de texto plano, en formato TSV (tab-separated values) de tres columnas, sin adorno de los literales de cadena.},
comments={El detalle de los formatos de fichero se encuentra en \url{https://www.kaggle.com/c/sentiment-analysis-on-movie-reviews/data}},
objasoc={\fullrefSRSObj{gui}},
reqasoc={\fullrefSRSIrq{traintsv}},
data={Análogo al \refSRSIrq{traintsv}, pero sin la última columna del sentimiento.},
priority=Alta,
stability=Alta}

\todo{IRQ: Formato del fichero de submisión}

\section{Requisitos funcionales}

\section{Requisitos no funcionales}

