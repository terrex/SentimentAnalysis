%!TEX root = pfc-memoria.tex
%!TEX encoding = UTF-8 Unicode

\chapter{Análisis de requisitos}

Este capítulo contiene la Especificación de Requisitos del Sistema (ERS en español, ó SRS del inglés \emph{System Requirements Specification})\index{ERS}\index{SRS} siguiendo las directrices de la norma IEEE~Std~830-1998 \citep{std830-1998} usando la notación de UML~2.5 \citep{UML2.5}

\section{Actores del sistema}

\SRSActor{title={Usuario},
label=usuario,
desc={Usuario normal de la aplicación, comúnmente un alumno en un puesto informático con la aplicación gráfica instalada, pantalla, procesador, teclado y ratón.},
comments={-/-},
stability=Alta}

\section{Objetivos}

\SRSObj{title={Biblioteca unificada de NLP y ML},
label=biblioteca-nlp-ml,
desc={Desarrollar una biblioteca unificada de NLP y ML para el análisis del sentimiento o polaridad de opiniones. La biblioteca debe proporcionar la funcionalidad necesaria para
\begin{enumerate}[a)]
\item procesar texto mediante diversos filtros (en la etapa de entrenamiento y la de clasificación),
\item aprender a clasificar el sentimiento automáticamente mediante algoritmos parametrizables de aprendizaje automático.
\end{enumerate}
Ambos grupos de funcionalidad deberán permitir la colaboración entre sí, en secuencia.},
comments={-/-},
stability=Alta}

\newpage
\SRSObj{title={GUI para el análisis de sentimiento},
label=gui,
desc={Desarrollar una aplicación de escritorio, con Interfaz Gráfica de Usuario multiplataforma, que haga uso del \refSRSObj{biblioteca-nlp-ml} para guiar al alumno en el análisis del sentimiento de opiniones sin necesidad de codificar nada.},
comments={-/-},
stability=Alta}


\section{Requisitos de información}

\section{Requisitos funcionales}

\section{Requisitos no funcionales}

