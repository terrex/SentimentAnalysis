%!TEX root = pfc-memoria.tex
%!TEX encoding = UTF-8 Unicode

\chapter{Procesamiento del Lenguaje Natural}

\section{Preprocesamiento}

\subsection{\emph{``Tokenización''}}

La \emph{tokenización}\index{tokenización}\index{tokenization@\emph{tokenization}} en NLP es normalmente el proceso de separar la cadena de entrada, frases en lenguaje natural, en las distintas palabras que componen la frase. Puede ser simplemente un separador que trocee la frase cada vez que se encuentre un espacio.
\begin{listing}[H]
\begin{minted}{python}
>>> 'of escapades demonstrating the adage that what is good for the goose'.split()
['of', 'escapades', 'demonstrating', 'the', 'adage', 'that', 'what', 'is', 'good', 'for', 'the', 'goose']
>>> 
\end{minted}
\caption{Tokenización sencilla mediante la separación de espacios.}
\label{lst:tokenizacion-sencilla}
\end{listing}

La tokenización en su sentido más amplio y general es el procedimiento del análisis léxico para reconocer lexemas del flujo de entrada con la que se obtiene un flujo de salida compuesto por tokens \citep[§2.2]{Jimenez2004}.

\begin{definition}[Token] \index{token}
Denominamos \emph{token} al conjunto de cadenas de caracteres con significado mínimo.
\end{definition}

\begin{definition}[Patrón] \index{patrón}
El \emph{patrón} es una regla que describe el conjunto de caracteres que cuadran un token.
\end{definition}

El concepto de ``cuadrar'' se denomina en inglés como match\index{match@\emph{match}}, y también se puede traducir como concordancia o correspondencia.

\begin{definition}[Lexema] \index{lexema}
Un \emph{lexema} es una secuencia concreta de caracteres que se corresponde con el patrón de un token determinado.
\end{definition}

\begin{example} \index{token} \index{patrón} \index{lexema}
Pongamos algunos ejemplos de tokens, patrones y lexemas, de un lenguaje formal típico de programación:
\nopagebreak
\begin{center}
\begin{tabular}{|l|l|l|}
\hline
\textbf{token} & \textbf{lexemas de ejemplo} & \textbf{descripción informal del patrón} \\ \hhline{===}
\codep{CONST} & \codep{const} & const \\ \hline
\codep{IF} & \codep{if} & if \\ \hline
\codep{OPREL} & \codep{<}, \codep{<=}, \codep{=}, \codep{>}, \codep{>=} & <~ ó <= ó = ó >~ ó >= \\ \hline
\codep{ID} & \codep{pi}, \codep{contador}, \codep{D2} & letra seguida de letras y dígitos \\ \hline
\codep{CTENUM} & \codep{3.1416}, \codep{0}, \codep{6.1E23} & cualquier constante numérica \\ \hline
\codep{LITERAL} & \codep{"core dumped"} & cualesquiera letras entre \codep{"} y \codep{"} excepto \codep{"} \\ \hline
\end{tabular}
\end{center}
\end{example}

En lenguaje natural, la unidad mínima de información textual será la palabra. De manera que token y lexema coinciden siempre, y no tiene sentido definir un patrón para cada token de las palabras de un diccionario de la lengua, ya que el patrón es la propia secuencia de letras de la palabra.

Existen distintos niveles de tokenización de documentos \citep[ch.~1]{Perkins2010}:
\nopagebreak
\begin{itemize}
\item Tokenizar textos en frases.
\item Tokenizar frases en palabras.
\end{itemize}

Simplemente para la primera tarea ya nos encontramos con dificultades, pues las frases no sólo terminan en punto. Pueden terminar en punto (\codep{.}), signo de exclamación (\codep[text]{!}), signo de interrogación (\codep[text]{?}), en algunos casos dos puntos (\codep{:}). Y en otros idiomas, como el español, también hay que tener en cuenta que en la separación entre frases puede haber signos de apertura de interrogación o exclamación, etc.

En cuanto a la separación de palabras, también pueden surgir problemas, dependiendo del idioma. En el siguiente ejemplo (\autoref{lst:wordtokenizers}) podemos ver distintas formas de separar palabras de una frase en inglés que usa contracciones.\footnote{También se puede encontrar un demostrador interactivo de la tokenización en la siguiente dirección:\\
\url{http://text-processing.com/demo/tokenize/}}

\begin{listing}[H]
\begin{minted}{python}
>>> from nltk.tokenize import TreebankWordTokenizer, WordPunctTokenizer, WhitespaceTokenizer
>>> TreebankWordTokenizer().tokenize("Can't is a contraction.")
['Ca', "n't", 'is', 'a', 'contraction', '.']
>>> WordPunctTokenizer().tokenize("Can't is a contraction.")
['Can', "'", 't', 'is', 'a', 'contraction', '.']
>>> WhitespaceTokenizer().tokenize("Can't is a contraction.")
["Can't", 'is', 'a', 'contraction.']
>>> 
\end{minted}
\caption{Diferentes estrategias de separación de palabras}
\label{lst:wordtokenizers}
\end{listing}

Una mejor aproximación al problema es reemplazar --previamente a la tokenización--, las palabras contraídas con sus palabras equivalentes sin contracción.\footnote{Puede obtenerse un catálogo bastante extenso de contracciones en inglés --en uso y arcaicas--, en la dirección: \url{http://www.enchantedlearning.com/grammar/contractions/list.shtml}}
Para ello es necesario construirse una tabla que establezca el mapping \citep{Perkins2014}.

Como ejemplo de un conjunto de substitución, véase una tabla no exhaustiva en la \autoref{tbl:contracciones}, en dos versiones:
\nopagebreak
\begin{enumerate}[(a)]
\item En la primera de ellas se implementaría empleando un diccionario común de palabras de búsqueda y reemplazo, directamente.
\item En la segunda opción, se hace uso de expresiones regulares para condensar la tabla y no tener que hacer explícito todas las posibles contracciones que compartan sufijos sinónimos.
\end{enumerate}

\begin{table}[htbp]
\centering
\begin{subtable}[t]{0.45\linewidth}
\centering
\begin{tabular}{|l|l|}
\hline
\textbf{contracción} & \textbf{reemplazar con} \\ \hhline{==}
isn't & is not \\ \hline
aren't & are not \\ \hline
wasn't & was not \\ \hline
weren't & were not \\ \hline
haven't & have not \\ \hline
hasn't & has not \\ \hline
couldn't & could not \\ \hline
shouldn't & should not \\ \hline
mightn't & might not \\ \hline
mustn't & must not \\ \hline
could've & could have \\ \hline
might've & might have \\ \hline
must've & must have \\ \hline
ma'am & madam \\ \hline
she'd've & she would have \\ \hline
'tisn't & it is not \\ \hline
who'll & who will \\ \hline
when'll & when will \\ \hline
what're & what are \\ \hline
we're & we are \\ \hline
they're & they are \\ \hline
that's & that is \\ \hline
I'll & I will \\ \hline
you'll & you will \\ \hline
he'll & he will \\ \hline
she'll & she will \\ \hline
\end{tabular}
\caption{Sustitución literal}
\label{tbl:contracciones-directa}
\end{subtable}
\begin{subtable}[t]{0.45\linewidth}
\centering
\begin{tabular}{|l|l|}
\hline
\textbf{contracción} & \textbf{reemplazar con} \\ \hhline{==}
\verb=r"won't"= & \verb="will not"= \\ \hline
\verb=r"can't"= & \verb="cannot"= \\ \hline
\verb=r"i'm"= & \verb="i am"= \\ \hline
\verb=r"ain't"= & \verb="is not"= \\ \hline
\verb=r"(\w+)'ll"= & \verb="\g<1> will"= \\ \hline
\verb=r"(\w+)n't"= & \verb="\g<1> not"= \\ \hline
\verb=r"(\w+)'ve"= & \verb="\g<1> have"= \\ \hline
\verb=r"(\w+)'s"= & \verb="\g<1> is"= \\ \hline
\verb=r"(\w+)'re"= & \verb="\g<1> are"= \\ \hline
\verb=r"(\w+)'d"= & \verb="\g<1> would"= \\ \hline
\end{tabular}
\caption{Sustitución condensada mediante expresiones regulares}
\label{tbl:contracciones-regexp}
\end{subtable}
\vfill
\caption{Sustitución de contracciones en inglés}
\label{tbl:contracciones}
\end{table}

\subsection{Radicación \emph{``Stemming''}}

La \emph{decodificación primaria}\index{decodificación primaria} es la herramienta instintiva que permite a los hablantes de un idioma intuir el significado de una palabra desconocida del idioma --siendo conocido el subconjunto común y más frecuente del mismo--, a partir de sus elementos léxicos \citep{Zubira2002}. Son tres:
\nopagebreak
\begin{description}
\item[La contextualización.]\index{contextualización} Por medio de este mecanismo rastreamos el significado de la palabra valiéndonos del contexto o entorno de las frases en las cuales está inscrita. Ejemplos:
\nopagebreak
\begin{itemize}
\item ``Lo admiran por su \textbf{sindéresis}, ya que su \emph{capacidad para juzgar} es notable.'' $\Longrightarrow$ El contexto nos permite deducir que el significado de sindéresis es la capacidad para juzgar.
\item ``Los \emph{derechos económicos, sociales y culturales} son de segundo orden. Los \textbf{DESC} suponen mayores demandas.'' $\Longrightarrow$ El contexto nos permite intuir que el acrónimo DESC se refiere a derechos económicos, sociales y culturales.
\end{itemize}
%
\item[La sinonimia.]\index{sinonimia} Mediante la sinonimia, se puede hacer coincidir el significado de la palabra desconocida con términos semejantes que se han mencionado antes de ella (anáfora) o que se mencionarán después (catáfora) en el texto. Ejemplo:
\nopagebreak
\begin{itemize}
\item ``Nadie ha visto al \textbf{abate}. Parece que este \emph{clérigo} ha desaparecido.'' $\Longrightarrow$ El sinónimo de abate es clérigo, aunque seguramente tenga algún matiz que lo convierte en un término con mayor especificidad.
\end{itemize}
%
\item[La radicación.]\index{radicación} En inglés \emph{stemming}\index{stemming@\emph{stemming}}. La radicación consiste en descomponer la palabra en sus elementos constituyentes o afijos: prefijos, infijos y sufijos. Ejemplos:\footnote{Pueden consultarse más ejemplos en \url{http://www.slideshare.net/sandracasierra/decodificacin-primaria}}
\nopagebreak
\begin{itemize}
\item \textbf{postdiluviano} $\Longrightarrow$ Después (post-) de un diluvio.
\item \textbf{subtropical}  $\Longrightarrow$ Por debajo (sub-) del trópico.
\end{itemize}
\end{description}

La manera más común de realizar la radicación\index{radicación}\index{stemming@\emph{stemming}} en aplicaciones de NLP es eliminar los sufijos para quedarse con la raíz de la palabra, y su significado más primitivo (véase \autoref{lst:PorterStemmer}).

\begin{listing}[htbp]
\begin{minted}{python}
>>> from nltk.stem import PorterStemmer
>>> stemmer = PorterStemmer()
>>> stemmer.stem('cookery')
'cookeri'
>>> stemmer.stem('cooking')
'cook'
>>> stemmer.stem('series')
'seri'
>>> stemmer.stem('women')
'women'
>>> stemmer.stem('riding')
'ride'
>>> 
\end{minted}
\caption{Funcionamiento de \codep{PorterStemmer}}
\label{lst:PorterStemmer}
\end{listing}

\codep{PorterStemmer} es la implementación en NLTK del algoritmo de radicación de Martin Porter \citep{Porter1980}. Es un algoritmo --muy usado para lengua inglesa--, de eliminación o reemplazo de sufijos en varias iteraciones.

Aunque se pierda algo de información, tiene sus ventajas:
\nopagebreak
\begin{itemize}
\item Sirve para reducir el tamaño del vocabulario; y por consiguiente, el orden del tiempo y el espacio que ocupan los programas de procesamiento.
\item Sirve para asemejar conceptos similares, que es normalmente lo deseable al hacer una búsqueda o extracción de información.
\item También sirve para acelerar el procesamiento y reducir el tamaño de los índices de los motores de búsqueda.
\end{itemize}

No obstante también tiene alguna desventaja:
\nopagebreak
\begin{itemize}
\item Al estar basado en la eliminación literal de sufijos, puede producir palabras no válidas del idioma (como en \emph{cookeri} en el ejemplo anterior).
\end{itemize}

NLTK proporciona otros stemmers adicionales a \codep{PorterStemmer}:
\begin{description}
\item[\codep{LancasterStemmer}] Desarrollado en la Universidad de Lancaster, es ligeramente más agresivo que el algoritmo de Porter, aunque no existen estudios que confirmen la supremacía de uno sobre el otro.
\item[\codep{RegexpStemmer}] Stemmer al que se puede recurrir para eliminar prefijos o sufijos de manera personalizada configurado mediante una expresión regular.
\item[\codep{SnowballStemmer}] Snowball extiende el método Porter para su aplicación a otros idiomas indoeuropeos: románicos, germánicos y escandinavos.
\end{description}

\subsection{Lematización}

La lematización\index{lematización} (\emph{lemmatizing}\index{lemmatizing@\emph{lemmatizing}} en inglés) es el proceso lingüístico que sustituye una palabra con forma \emph{flexionada} (plural, femenino, verbo conjugado, \ldots) por su \textbf{lema}. Tiene el mismo objetivo que la radicación, pero a diferencia de aquel, la lematización siempre produce una palabra válida en el idioma.
\begin{definition}[Lema]
En lingüística, el \emph{lema}\index{lema} es la forma que por convenio se acepta como representante de todas las formas flexionadas de una misma palabra.
\end{definition}

\chapter{Aprendizaje automático}

\section{Clasificadores}

\subsection{Bondad del clasificador (valor-F)}

Para la comparación de sistemas de búsqueda y recuperación de información disponemos de una métrica llamada valor-F\index{valor-F} (\emph{F-score}\index{F-score@\emph{F-score}}, \emph{F-measure}\index{F-measure@\emph{F-measure}} ó \emph{$F_1$ score}\index{F\_1 score@$F_1$ score}) siendo ésta la media armónica de otros dos indicadores \citep{wikipedia:precision-exhaustividad}:
\begin{description}
\item[precisión \emph{(precision)}] \index{precisión}\index{precision@\emph{precision}}
Es la fracción de instancias recuperadas que son relevantes.
\begin{eqnarray}
\text{precisión} &=& \frac{|\{\text{relevantes}\}\cap\{\text{recuperados}\}|}{|\{\text{recuperados}\}|}
\end{eqnarray}
\item[exhaustividad \emph{(recall)}] \index{exhaustividad}\index{recall@\emph{recall}}
Es la fracción de instancias relevantes que han sido recuperadas.
\begin{eqnarray}
\text{exhaustividad} &=& \frac{|\{\text{relevantes}\}\cap\{\text{recuperados}\}|}{|\{\text{relevantes}\}|}
\end{eqnarray}
\item[valor-F \emph{(F-score)}] Media armónica de precisión y exhaustividad.
\begin{eqnarray}
F_1 &=& 2\times\frac{\text{precisión}\times\text{exhaustividad}}{\text{precisión}+\text{exhaustividad}}
\end{eqnarray}
\end{description}

Supongamos un clasificador binario:
\begin{itemize}
\item La medida de precisión será el porcentaje de documentos correctamente clasificados dentro del conjunto de prueba.
\item La medida de exhaustividad será el porcentaje de documentos del conjunto de referencia que han sido correctamente clasificados. \citep{Perkins2010}
\end{itemize}


\section{Rendimiento en Python}

Se observó que la carga del modelo entrenado \path{GoogleNews-vectors-negative300.bin.gz} consumía mucho tiempo y espacio ($\approx$\si{4.5}{GiB} y unos 5 minutos). Este modelo ha sido publicado por Google en el repositorio del proyecto \codep{word2vec} basado en el \emph{dataset} de Google News (aproximadamente 100 mil millones de palabras). El modelo contiene vectores 300-dimensionales para 3 millones de palabras y de frases (bigramas y trigramas). Las frases se obtuvieron usando una aproximación dirigida por datos sencilla, como se encuentra descrito en \cite{DBLP:journals/corr/MikolovSCCD13}.

Por ello se convirtió inicialmente el modelo del \emph{Word2Vec} original en la representación interna provista por \codep{gensim}, que soporta en \emph{unpickling} de datos de NumPy, de mejor desempeño.

El procedimiento para la conversión es:

\begin{listing}[H]
\begin{minted}{python}
from gensim.models.word2vec import Word2Vec
# lectura sin optimizar
model_orig = Word2Vec.load_word2vec_format('GoogleNews-vectors-negative300.bin', binary=True)
# escritura optimizada
model_orig.save('GoogleNews-vectors-negative300.bin.gensim')
\end{minted}
\caption{Conversión del formato crudo \codep{word2vec} en el optimizado por \codep{gensim}}
\label{lst:word2vec-convert}
\end{listing}

Y para la carga de la versión optimizada:

\begin{listing}[H]
\begin{minted}{python}
from gensim.models.word2vec import Word2Vec
# lectura optimizada
model = Word2Vec.load('GoogleNews-vectors-negative300.bin.gensim', mmap='r')
\end{minted}
\caption{Lectura del modelo optimizado por \codep{gensim} previamente almacenado}
\label{lst:word2vec-convert}
\end{listing}
