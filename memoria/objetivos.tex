%!TEX root = pfc-memoria.tex
%!TEX encoding = UTF-8 Unicode

\chapter{Objetivos}

\epigraph{``Si tú no construyes tus sueños, alguien te contratará para construir los suyos.''}{\textsc{Tony Gaskins} (1984--)}

\section{Objetivos generales}

Los objetivos del proyecto son los siguientes:

\begin{itemize}
\item Estudiar las características del lenguaje natural que lo hacen diferente del lenguaje formal.
\item Estudiar algunas estrategias, métodos, procedimientos o algoritmos para analizar y extraer información de textos en lenguaje natural.
\item Estudiar algoritmos de aprendizaje automático aplicado a tareas de NLP.
\item Desarrollar una biblioteca unificada de NLP y aprendizaje automático para el análisis del sentimiento o polaridad de opiniones.
\item Desarrollar un asistente con una Interfaz Gráfica de Usuario (IGU, \emph{GUI})\index{IGU}\index{GUI} para usar en el aula.
\end{itemize}

\section{Alcance}

El proyecto está orientado a introducir al alumno en las técnicas de procesamiento de lenguaje natural para tareas de clasificación de documentos de texto.

Se presuponen conocimientos básicos de:
\begin{itemize}
\item álgebra
\item estadística
\item programación orientada a objetos
\item teoría de autómatas y lenguajes formales
\item procesadores de lenguajes
\end{itemize}
de las asignaturas de la titulación.

