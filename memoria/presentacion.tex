%!TEX TS-program = xelatex
%!TEX encoding = UTF-8 Unicode
%!BIB TS-program = biber
%!BIB program = biber

\documentclass[xetex,compress,spanish,tikz]{beamer}

\setbeameroption{show notes}
%\setbeameroption{show only notes}
%\setbeameroption{hide notes}

\mode<presentation>
{
  \usetheme{Warsaw}
  \usecolortheme[rgb={.7,.2,.2}]{structure}
  \setbeamercovered{transparent}
}
\useoutertheme[footline=authortitle,subsection=false]{miniframes}

\usepackage[no-math]{fontspec}
\usepackage[spanish]{babel}
\usepackage{xltxtra}
\usepackage{fontspec}
\usepackage{xunicode}
\usepackage{amsmath}
\usepackage{amsthm}
\usepackage{graphicx}
\usepackage{amssymb}
\usepackage{hyperref}
\usepackage{siunitx}
\usepackage{tikz}
\usetikzlibrary{calc}

%% bib %%
\usepackage[style=authoryear,natbib=true,%
maxbibnames=99,maxcitenames=2,%
citestyle=authoryear-comp,doi=true,url=true,backend=biber,dashed=no]{biblatex}
\bibliography{\jobname}
%% style %%
\defaultfontfeatures{Mapping=tex-text,Numbers={OldStyle}}
\setmainfont[Mapping=tex-text]{Hoefler Text}
\setromanfont[Mapping=tex-text]{Hoefler Text}
\setsansfont[Scale=MatchLowercase,Mapping=tex-text]{Gill Sans}
\setmonofont[Scale=0.9]{Courier New}
\sisetup{output-decimal-marker={,},
	product-units=single,
	detect-all, detect-inline-family=text, detect-inline-weight=text,
  detect-display-math=true}

%% custom macros %%
\newcommand{\email}[1]{%
  \href{mailto:#1}{\nolinkurl{<#1>}}}

\DeclareSIUnit[number-unit-product = {\,}]
	\pixel{px}

%% doc info %%
\title{Procesamiento del Lenguaje Natural Aplicado al Análisis del Sentimiento de Opiniones}
\author[Guillermo Gutiérrez-Herrera]{Proyecto Fin de Carrera\\Ingeniero en Informática (Plan 97)\\[1em]
Realizado por: Guillermo Gutiérrez-Herrera\\[1em]
Dirigido por: José Antonio Troyano}
\institute{Departamento de Lenguajes y Sistemas Informáticos\\
Escuela Técnica Superior de Ingeniería Informática\\
Universidad de Sevilla}
\date{Septiembre de 2015}
\logo{\includegraphics[width=10pt]{logo-lsi}\hspace{2pt}\includegraphics[width=15pt]{logo-us}}
\subject{Procesamiento de NLP aplicado al Análisis de Sentimiento}
\keywords{Natural Language Processing, Machine Learning, Sentiment Analysis}

%% documento %%
\begin{document}

\frame{
\titlepage

\tikz[overlay,remember picture]
\node[anchor=center] at ($(current page.south west)+(1.3,1.3)$) {
\includegraphics[width=40pt]{logo-us}
};

\tikz[overlay,remember picture]
\node[anchor=center] at ($(current page.south west)+(2.9,1.3)$) {
\includegraphics[width=30pt]{logo-lsi}
};

\note{[Buenos días] / [Buenas tardes]. Mi nombre es Guillermo Gutiérrez y voy a presentar mi Proyecto Fin de Carrera titulado \emph{Procesamiento del Lenguaje Natural Aplicado al Análisis del Sentimiento de Opiniones} y dirigido por el profesor José Antonio Troyano.}
}

%\section[Índice]{}
\frame{\frametitle{Índice}
\tableofcontents

\note[item]<1>{Empezaremos la presentación con una breve introducción y los objetivos del proyecto; y la planificación y metodología utilizados para su realización.}
\note[item]<1>{En la siguiente parte resumiremos los métodos de Procesamiento de Lenguaje Natural y las técnicas de Aprendizaje Automático que se describen en la memoria.}
\note[item]<1>{A continuación presentamos el problema concreto de clasificación automática del sentimiento; y el diseño de la solución propuesta.}
\note[item]<1>{Y por último las conclusiones del proyecto y la bibliografía que se ha usado principalmente para la realización de este proyecto.}
}

\section[Introducción]{Introducción y objetivos}

\frame{\frametitle{Introducción (I)}
\begin{itemize}
\item Los compiladores, traductores y procesadores de lenguajes tradicionales procesan \textbf{lenguajes formales.}
\item El lenguaje formal se diseña formalmente para evitar la ambigüedad.
\item El \textbf{lenguaje natural} es ambiguo por naturaleza.
\item Dentro del ámbito de la Inteligencia Artificial, se ha comprobado eficaz aplicar Aprendizaje Automático al tratamiento del lenguaje natural.
\item En este proyecto
\begin{itemize}
\item se resumen estas técnicas
\item se desarrolla una aplicación gráfica para usar como laboratorio de experimentación
\end{itemize}
\end{itemize}
}

\section[Planificación]{Planificación y metodología}

\section[NLP]{Procesamiento del Lenguaje Natural}

\section[ML]{Aprendizaje Automático}

\section{Problema}

\section[Diseño]{Diseño de la solución}

\section[Conclusiones]{Conclusiones y continuidad}

\section{Bibliografía}

\appendix

\frame{
\Huge ¿Preguntas?
\tikz[overlay,remember picture]
\node[anchor=center] at ($(current page.center)+(2,0)$) {
  \includegraphics[width=120pt]{preguntas}
};
}

\frame{
\centering
\huge Gracias por su atención

\vspace{0.2\textheight}

\normalsize
\begin{minipage}{5.3cm}
Guillermo Gutiérrez-Herrera

\email{guiguther@alum.us.es}

\email{xiterrex@gmail.com}
\end{minipage}
}

\end{document}
